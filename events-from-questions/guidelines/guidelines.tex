% Created 2013-10-30 Wed 16:31
\documentclass[11pt]{article}
\usepackage[utf8]{inputenc}
\usepackage[T1]{fontenc}
\usepackage{fixltx2e}
\usepackage{graphicx}
\usepackage{longtable}
\usepackage{float}
\usepackage{wrapfig}
\usepackage{soul}
\usepackage{textcomp}
\usepackage{marvosym}
\usepackage{wasysym}
\usepackage{latexsym}
\usepackage{amssymb}
\usepackage{hyperref}
\tolerance=1000
\usepackage{color}
\usepackage{listings}
\usepackage{geometry}
\providecommand{\alert}[1]{\textbf{#1}}

\title{Creating a Corpus of Questions and Answers about Biological Processes: Annotation Guidelines}
\author{Jonathan Berant and Vivek Srikumar}
\date{}
\hypersetup{
  pdfkeywords={},
  pdfsubject={},
  pdfcreator={Emacs Org-mode version 7.9.2}}

\begin{document}

\maketitle



\section{Introduction}
\label{sec-1}

  
  We have the ability to read text that describes a biological process
  (that is, a collection of inter-connected events that lead to an end
  result) and answer complex questions about the relationships between
  the events. Our goal is to develop systems that can automatically
  answer complex biology AP style questions in such a reading
  comprehension setting. We will use a hand created corpus of
  questions associated with text to train and evaluate the systems.
\section{Generating questions and answers}
\label{sec-2}

  
  The goal is to generate multiple-choice questions about biological
  processes that are described in a paragraph of text. The questions
  should focus on the events and entities participating in the
  process. Consider the following paragraph (from the textbook
  \emph{Biology} by Campbell and Reece) as an example:

\begin{quote}
The light reactions are the steps of photosynthesis that convert solar
energy to chemical energy. Water is split, providing a source of
electrons and protons (hydrogen ions, $H^+$) and giving off O2 as a
by-product. Light absorbed by chlorophyll drives a transfer of the
electrons and hydrogen ions from water to an acceptor called $NADP^+$
(nicotinamide adenine dinucleotide phosphate), where they are
temporarily stored. The electron acceptor $NADP^+$ is first cousin to
$NAD^+$, which functions as an electron carrier in cellular
respiration; the two molecules differ only by the presence of an extra
phosphate group in the $NADP^+$ molecule.
\end{quote}

  There are several events described in the paragraph -- splitting of
  water, absorption of light, transfer of electrons and hydrogen ions,
  etc. These events involve entities like water, electrons and
  protons, chlorophyll, etc. 

  We can write several questions about these. Some examples are listed
  below, with the correct answer marked in \textbf{boldface}:
  
\begin{enumerate}
\item A source of electrons and protons are provided after which
     event?
\begin{enumerate}
\item \textbf{Water is split}
\item Light is absorbed
\end{enumerate}
\item Which of the following events is caused by the absorption of
     chlorophyll?
\begin{enumerate}
\item \textbf{Transfer of electrons and protons into} $NADP^+$
\item The splitting of water
\end{enumerate}
\item What event would not happen if water does not provide electrons
     and hydrogen ions?
\begin{enumerate}
\item Light absorption by chlorophyll
\item \textbf{Transfer of ions to} $NADP^+$
\end{enumerate}
\end{enumerate}
\section{Guidelines for generating questions and answers}
\label{sec-3}

  We are primarily interested in questions that depend on the
  inter-relationships between events. An event can be a \emph{subevent} or
  a \emph{super-event} of another one. Additionally, an event can
  \emph{enable}, \emph{cause} or \emph{prevent} another one. Note that
  these \textbf{event-event} relations often imply a temporal ordering
  between them. For example, if event e$_1$ causes an event e$_2$, then
  e$_1$ should occur before e$_2$.

  Entities can play different roles in one or more events. For
  example, an entity can be the performer of an event (or more
  formally, the \texttt{Agent} of the event), or it can be acted upon in the
  event (that is, the \texttt{Patient} of the event), it could be generated
  in the event (the \texttt{Result} of the event), and so on.

  We have identified the following classes of questions that verify
  understanding of these relationships between events and entities:

\begin{enumerate}
\item For event $e$:
\begin{enumerate}
\item What event will be caused or prevented by $e$?
\item If $e$ does not happen, what else will not happen?
\item What event \emph{should} occur after/before $e$?
\item What events are necessary for $e$ to occur?
\end{enumerate}
\item For events $e_1$, $e_2$:
\begin{enumerate}
\item Which one happens first?
\item What is the relation between them (eg. e$_1$ causes e$_2$, e$_1$ is
        a super event of e$_2$, and so on)?
\item What is the sequence of events between them?
\end{enumerate}
\item For events e$_1$,\ldots{},e$_n$:
\begin{enumerate}
\item What is the correct ordering of the events?
\item Which may simultaneously occur?
\end{enumerate}
\item Which entity performs a given role (eg. \texttt{Agent}, \texttt{Theme},
     \texttt{Result}) for an event?
\item What role does an entity perform in an event?
\item For entity $a$:
\begin{enumerate}
\item What entities are necessary to produce $a$?
\item What events are necessary to produce $a$?
\item If $a$ is not produced what events will not happen?
\item If $a$ is not produces what other entities will not be
        produced?
\end{enumerate}
\item If a$_1$ and a$_2$ are two entities in the process, how does a$_1$ lead
     to the production of a$_2$?
\end{enumerate}

  Note that these are only templates for types of questions and the
  actual questions need not look like them. For example, the first
  question in the three example questions listed above asks what
  events are necessary to produce an entity (template 6.2). Similarly,
  the second question belongs to the template 1.1 and the third one
  belongs to the template 1.2.
\subsection{Other guidelines}
\label{sec-3-1}

\begin{enumerate}
\item Each question \textbf{should} belong to one of the templates
      identified above.
\item Each question should be associated with two answers, where only
      one is unambiguously correct and the other is unambiguously
      incorrect.
\item It should be possible to answer the question by reasoning about
      the events and entities and their relationships, as specified in
      the text.
\item Do not use background knowledge that is not present in the
      text. In the above example, if the text did not identify that
      protons are hydrogen ions, represented by $H^+$, we should not
      use these names in the questions or the answers.
\item When referring to entities and events in the questions and
      answers, try to use their names as they appear in the paragraph.
      However, sometimes the same entity may be referred to by
      different names (like proton or $H^+$ in the paragraph). If
      (and only if) this happens, you can refer to the entity by any
      of these names.
\item Do not use contractions or drop words in the names of entities
      unless the text becomes awkward without doing so.
\item We are only interested in events and entities that participate
      in them. In the paragraph above, the last sentence says ``the two
      molecules differ only by the presence of an extra phosphate
      group in the $NADP^+$ molecule''. Note that this sentence does
      not describe an event. Do not generate questions based on such
      sentences.
\end{enumerate}

   

\end{document}
