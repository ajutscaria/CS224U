\citeauthor{chju2008ec} classifies the temporal relations between two events with an SVM model using local features pertaining to the events themselves. \citeauthor{temporalordering} also uses this kind of model and stick to classifying only the before/after relation between events, but the latter also tries to overcome some of the global problems associated with localized event ordering algorithms. However they classify the events, without identifying association rules between events and their absolute time of occurence like the work of \citeauthor{quang}.

The three papers add up gradually on work in the area of classification of temporal relations between two events. The timeline representation for ordering events has some advantages over temporal graph representations in the former papers. The timeline model allows events to be associated with precise time intervals, which improves human interpretability of the temporal relations between events and time. It also simplifies global inference formulations as the number of variables and constraints needed in the ILP is more concise relative to time point based formulation.  

\citeauthor{temporalordering} assumes the event-time relation to be given and tries to improve event-event temporal relation classification, whereas the model of \citeauthor{quang} jointly optimizes both tasks at the same time. For local classifiers, it can be seen that all three papers use similar types of features and the latter two encode transitive closure of relations between event mentions within the global inference model. Also, the two later models use Integer Linear Programming for performing the joint inference with a set of global constraints that enforce global coherency and this is seen to be better than the greedy strategy of adding pairs of events one at a time, ordered by their confidence. The comparison of classifiers also shows that \citeauthor{quang} could improve the accuracy of \citeauthor{temporalordering} to more than 5%.
