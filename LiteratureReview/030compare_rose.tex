The three papers \citeauthor{chju2008ec}, \citeauthor{temporalordering}, and \citeauthor{quang} incrementally extend the work in the area of classification of temporal relations between two events. Some of the main areas common to these papers are as follows:

\begin{enumerate}
\item \xhdr{Types of relations amongst events} \citeauthor{chju2008ec} uses {\em before} against {\em others}, \citeauthor{temporalordering} adds {\em after} and \citeauthor{quang} adds {\em overlap} to this set of features.

\item \xhdr{Local classifier} \citeauthor{chju2008ec} classifies the temporal relations between two events with an SVM model using local features pertaining to the events themselves. The SVM has two stages, one to label temporal attributes of events, and a  second stage to classify the temporal relationship between the two events. \citeauthor{temporalordering} also uses this kind of a model whereas \citeauthor{quang} uses  regularized average Perceptron, all three papers using similar features for classification. However it is not clear if one of these is more advantageous than the other. Also, \citeauthor{temporalordering} assumes the event-time relation to be given and tries to improve event-event temporal relation classification, whereas the model of \citeauthor{quang} jointly optimizes both tasks at the same time.

\item \xhdr{Extending event relationships in graph} Transitivity rules were applied to extend the labeling, but it was seen to not improve the results. This was assumed in \citeauthor{chju2008ec} to be because some of the documents contained inconsistent labels which created poor transitivity additions. But the second paper overcomes this problem by adding time-time information to the model, which increased the density of the graph by adding relations based on this new information as well. This turned out to be very beneficial. 

\item \xhdr{Timescale} The timeline representation for ordering events in \citeauthor{quang} has some advantages over temporal graph representations in the former papers. The timeline model allows events to be associated with precise time intervals, which improves human interpretability of the temporal relations between events and time. It also simplifies global inference formulations as the number of variables and constraints needed in the Integer Linear Programming is more concise relative to time point based formulation. The comparison of classifiers also shows that \citeauthor{quang} could improve the accuracy of \citeauthor{temporalordering}, so using time intervals could be useful.

\item \xhdr{Joint classifier} The two later models use ILP for performing the joint inference with a set of global constraints that enforce global coherency and this is seen to be better than the greedy strategy of adding pairs of events one at a time, ordered by their confidence. This was also useful in avoiding some of the obvious mistakes made due to misclassification by the local classifiers.

\end{enumerate}
