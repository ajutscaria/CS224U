\xhdr{1. Unsupervised Learning of Narrative Event Chains - \citeauthor{chju2008ec}}

This paper mainly talks about unsupervised event identification and temporal event ordering. The authors follow the intuition that events related to a common protagonist (the common argument for the event verbs) should form a valid narrative event chain. They do not pay much attention to the type or role of the protagonist when grouping events with a common protagonist together. Hence, no heuristic involving the role and type of the protagonist is used to get a richer event extraction. This paper constructs a novel scoring method for predicting the most likely next event in a chain that involves two kinds of scores - local pairwise event scores and global narrative chain scores.

The second major goal in the paper was to perform temporal ordering of events. The first stage for this was to use a temporal classifier that would label the temporal attributes of the events with tense and grammatical aspect information. In the second stage, features such as event-event syntactic properties, bigram event tense, bigram grammatical aspects and sentence location of the events are used to classify the temporal relationship between two events. The paper just explores the establishment of the {\em before} relationship in case of ‘temporal order identification’. Finally, to compare the outcome of event identification and temporal ordering, with ‘scripts’ (supervised hand labeled methods for event extraction), the paper suggests discretizing the narrative event chains. The discretization can be done by first performing agglomerative clustering of events based on the scores discussed above and then ordering these clusters by the temporal ordering method.
Thus the main takeaways from this paper are intuitions behind performing event identification, temporal ordering and finally merging these two to generate script-like structures as event extractions (all by unsupervised methods).

\xhdr{2. Unsupervised Learning of Narrative Schemas and their Participants - \citeauthor{chju2009}}

This paper addresses narrative event chains in further detail. In the broader perspective, it addresses the areas of ‘event identification’ and ‘argument extraction’ listed above. Moving on from identifying single event chains, they talk about overlapping these single chains to form a ‘narrative event schema’. Argument extraction is also done with unsupervised semantic role labeling (without the need of a predefined class of roles or hand built domain knowledge). These two tasks also contribute to the improvement of each other since rich event extraction largely revolves around extracting correlated events by identifying coreferring arguments and linking them. 

The semantic role labeling task that the paper takes up, is done in an unsupervised manner and by learning the roles automatically rather than picking them from a pool of predefined domain of roles. Once these labels are assigned, they can then construct narrative event chains, where the events have arguments with a type and role.  They suggest that the types may also be picked by finding the most frequent head-word in co-referential chains for the arguments.

If narrative chains are formed only based on the frequency with which two verbs share their arguments, ignoring the features of the arguments themselves, then it is likely that word sense ambiguities cause a comparatively worse event to be clustered with other events. Instead the paper recommends having the aforementioned attached types and roles for the arguments and modeling argument overlaps across all (event, argument) pairs to extract most likely next event in the narrative schema. Using this richer representation of event chains, they are combined into narrative schemas by scores depicting chain similarities and narrative similarities.\\

\xhdr{3. Template-Based Information Extraction without the Templates - \citeauthor{chju2011}}

The main aim of this paper is to devise a novel approach to information extraction, which uses methods to induce templates (structures which have slots, events and arguments to describe raw textual information). This links to our topic of event extraction since, template based information extraction can easily be extended to event extraction as templates and event structures are often similar. The major novelty of the paper lies in the method introduced in the paper to learn the template structure from an unlabeled corpus. The main positive of this paper over other approaches is that redundant documents about specific events are not needed to perform template induction. Also, templates with any type and any number of slots can be filled in an unsupervised fashion. After this template induction, event extraction draws from these templates.

The main task described is to extract information from a domain specific corpus by learning a rich template structure. The method for this involves
\begin{itemize}
\item Clustering the event patterns in the domain to estimate template topics using LDA and agglomerative clustering based on word distance and PMI over all event patterns. 
\item Building a new corpus specific to each cluster by retrieving documents from a larger unrelated corpus.
\item Inducing semantic role labels for each of the slots in the template using the larger corpus of documents.
\end{itemize}

The paper then talks about evaluation using two metrics - overall template evaluation as well as a stricter per template evaluation. Moreover, the results have comparable precision and an F1 score that approaches existing algorithms which rely heavily of prior knowledge of the domain.

Since this approach is unsupervised, the recall is hurt. Also since event extraction here happens as an after stage of template induction, the number of parameters required are high and could be reduced in future work. Other than pointing out future prospects for these improvements, in the larger picture the paper draws out an interesting way of event extraction as a step on top of template induction, where the latter is done in an unsupervised manner for learning the templates, slots and the roles.\\
