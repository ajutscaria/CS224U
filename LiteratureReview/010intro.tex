The web has an enormous amount of information stored as text content. While this information is accessible to anyone with an internet connection, analyzing this vast amount of data to extract meaningful information is rather difficult because of its sheer volume. Designing automated mechanisms to parse the content of the web to build automatic question answering systems has been an area of research for many years. For instance, extracting events and the associated entities from biomedical and molecular biology text has drawn a lot of attention, especially because of the introduction of BIONLP tasks targeting fine-grained information extraction. This has also been motivated by the increasing number of electronically available publications stored in databases such as PubMed. 

The task of event and entity extraction encompasses several areas including natural language processing, computational linguistics and text mining. This task would in turn help other popular NLP tasks such as question answering and machine translation. Different methods that make use of syntactic and dependency parsing, optimization techniques, semantic role labeling and machine learning techniques have been used over the years to tackle this challenging problem. While some of the systems built have had significant improvement over the previously existing ones, the performance of the state-of-the-art systems is still quite distant to that of a 'perfect' extraction system. This is mostly because interpreting meaning from complex structures of natural language is a hard task. In this paper, we review nine research publications from the area of event and entity extractions and temporal ordering of events. In the first section, we summarize the content of the literature reviewed and highlight their main design characteristics and contributions. Then, we compare the different approaches taken in the papers to compare and contrast the different strategies that have been adopted, their similarities and differences.