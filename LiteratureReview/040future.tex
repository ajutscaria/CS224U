In this paper, we investigated in detail some of the models that have been designed for event and argument extraction, and temporal ordering of events. Even though a lot of work has been done in these fields, we feel there is a lack of focus on extending these models to building automatic question answering systems based on the events extracted. For instance, there is no system, when given a textbook in Biology as input, extracts different high level processes involved and is capable of answering questions that involve the different events, its prerequisites, entities involved, output, and its temporal relation to other events. This could be probably be done by combining works discussed in this literature review. One could start by finding events and associated attributes, to build event chains that... After finding these main events, temporal ordering of these is required to determine the order of happening of each event, so one could say which event precedes another event. It might be good to use time intervals to associate each event with a particular interval so that merging different events onto a single timeline would be easy and this would help in finding answers that link related events from different parts of a text book.

Developing such a system would find application in a multitude of areas in this age of information explosion. To start with, it could help students to look for information in text books or encyclopedias without having to go through them page by page. There is also a huge amount of data available in the web, but unless we have an efficient event extraction system, most of it will lie unusable. Of course, we cannot rely fully on the data available from the web, but again, we can use different heuristics to build confidence on the data extracted as we see more supporting evidence, for instance.