NOT LATEST FILE

The work by \citeauthor{bjorne} performs event extraction by a pipeline of three independent steps of identifying triggers first, followed by argument extraction and then semantic prost processing. This method  is prone to cascading errors introdcued in early stages of the pipeline. For instance, if a trigger is missed in the first stage, we will never be able to extract the full event that it results in. Even though this can be tackled by passing several additional candidate to the next stage, this will increase the false positive rate as highlighted by \cite{miwa2010c}. In addition, these models cannot make use of the rich set of features by looking at how different events and entities interact with each other. In addition, the different rule based post-processing steps that need to be used to clean up the event-entity combinations extracted might lead to partially correct relations to be thrown out because of errors introduced in some stage of the pipeline. In short, joint models helps to capture the dependencies better and are more robust.

Both \citeauthor{bjorne} and \citeauthor{riedelmc} use graph structures to encode the event-entity and event-event relationships, although in different ways. In \citeauthor{bjorne} event extraction is done disjointly and NER nodes (known) and nodes for triggers(predicted by a classifier) are joined by edges. \citeauthor{riedelmc} uses graphs to generate binary variables as edges denoting entity-entity relationships. But, they solve the problem of event extraction as an optimization problem over binary variables.

Also, \citeauthor{bjorne}, \citeauthor{chju2008ec} and \citeauthor{riedelmc} have the entities taking part in events already provided to them and focuses on event extraction, while \citeauthor{toutanova} concerns with semantic role labeling and models argument prediction jointly with event prediction task by building an ensemble of local and global classifiers.

\citeauthor{riedelmc} uses the parse tree structure quite extensively for the task of semantic role labeling. Whereas the other models for event extraction doesn't use much of syntactic information. For instance, the re-ranking model in \citeauthor{riedelmc} maps the features from the parse tree directly into vector space. This helps them use more features relating to ordering and hierarchy of words.
