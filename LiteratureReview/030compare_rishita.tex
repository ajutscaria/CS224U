%Broad intro
In our literature review, we cover research papers that broadly fall into two categories. One category deals with event extraction and semantic role labeling including literature that covered biological event extraction for the BIONLP task.The other deals with temporal ordering of events.

%Intro to Rishita's section
The first series of three papers that we chose(\citeauthor{chju2008ec}, \citeauthor{chju2009}, and \citeauthor{chju2011}) broadly deals with one or more of - 'event identification', 'temporal ordering' or 'argument extraction' which they identify as the main parts of event extraction. Narrative event chains, schemas and template based event extractions are the topics that emerge out of these papers for event extraction.

Some important points which come up while comparing papers in this series are:
\begin{enumerate}
\item \xhdr{Cardinality of arguments} \citeauthor{chju2008ec} discusses chaining of events based on just one argument or participant. \citeauthor{chju2009} builds up on that by representing other entities involved in the events as well. This information of recognizing multiple entities can be very valuable, since now it is possible to overlap various event chains and form a larger event scenario or a 'narrative event schema'. The third paper, \citeauthor{chju2011}, also performs unsupervised template learning where the number of slots in the templates is inferred automatically without hand labeling or supervision. 

\item \xhdr{Argument Types and Roles} \citeauthor{chju2008ec} also does not pay much attention to the ‘type’ or ‘role’ of the protagonist. Role information such as knowing if the protagonist is a place or a person or an object could help in clustering the events more richly as well. More than mere verb comparison based on exact same argument, role and type information given to arguments in \citeauthor{chju2009} leads to a stronger approach for understanding the event chains and discarding certain candidates while finding the next most likely event. Template based event extraction in \citeauthor{chju2011} also talks of inferring semantic roles for the slots in the template prior to event extraction from the templates.

\item \xhdr{Approaches to Argument Extraction} In the third paper in the series, \citeauthor{chju2011}, event extraction here is done one step after template induction. Although this is a very systematic approach for information extraction without specific requirements, it might be overkill to use a learning approach with so many parameters and template induction for just event extraction. On the other hand, the first two papers, as mentioned above either use an exact match for arguments (protagonist method) or use semantic role labeling on arguments, without forming anything along the lines of a template.

\item \xhdr{Clustering Event Patterns} An important similarity in papers \citeauthor{chju2008ec} and \citeauthor{chju2011} is their use of agglomerative clustering for clustering events. In the first paper, event chains are discretized to be comparable with {\em hand labeled scripts} using agglomerative clustering on the pairwise scores. Along similar lines, \citeauthor {chju2011} talks of clustering events by agglomerative clustering before the event clusters are used to fetch larger corpora and induce template slot roles. This gives us a good idea of how clustering events can be useful to discretize events and partition domains for a higher level perspective.
\end{enumerate}
 
