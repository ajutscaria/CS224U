\section{Related Work}

BioNLP work

Timeline construction work.

Scripts work - Chambers, Poon 2013.

Work that uses global constraints with ILP or dual decomposition or whatever.

Process extraction is related to two recent lines of work in Information Extraction -- event extraction and timeline construction. The BioNLP 2009 and 2011 shared tasks \cite{kim09,kim11} led to increasing interest in biomedical event extraction \cite{Poon10,Miwa10,riedel11fast,Mcclosky11,Bjorne11}, where given a single sentence annotated with protein mentions, events are identified and relations between events and proteins are extracted. In this shared task participants were asked to consider nine event types that are relevant for proteins (such as \emph{Phosphorilation} and \emph{Transcription}). Processes, on the other hand,  are centered around discovering \emph{relations} between events that span \emph{multiple} sentences. In Figure~\ref{fig:process} for instance, process extraction involves determining the relations between 8 events (\emph{flowing}, \emph{enter}, etc.) that are necessary for ATP production, which appear across four sentences. Note that the set of possible event types can not be restricted to a small closed set.

Timeline construction involves identifying temporal relations between events \cite{Chambers08,Yoshikawa09,Denis11,Do12,Mcclosky12}, and is thus related to process extraction as both focus on event-event relations that span multiple sentences. However, fully capturing process structure requires handling a rich set of relations such as \textsc{Causes} and \textsc{SuperEvent} (see Section~\ref{sec:process}), which are often not addressed in timeline construction. Moreover, processes exhibit particular properties that do not hold generally in temporal ordering. For example, in processes all events are somehow related to one another, a property that can be exploited for improving extraction.
 

