\section{Introduction}

Biological processes are complex phenomena that involve a series of events and multiple participants. 
The paragraph in Figure~\ref{fig:process}, for example, describes a biological process of ATP synthesis.
This process involves 15 entities and 10 events. On top of that,  it describes the relationship between entity mentions (e.g., the two \textit{H+ ions} mentions are the same entity), role of an entity in an event (e.g.  \textit{H+ ions} is an \textit{agent} of event \textit{flowing}), and relationship between events (e.g., the second occurrence of \textit{enter} event \textit{causes} the \textit{changing} event). 

Automatically extracting the structure of processes from text is a necessary step for applications that require reasoning over such process events. 
A human reading this example paragraph can answer questions such as:
\begin{enumerate}[itemsep=0pt,topsep=0pt] 
\item \footnotesize \emph{How do H+ ions contribute to the production of ATP?}
\item \footnotesize\emph{What causes the rotor to spin?}
\item \footnotesize \emph{In ATP synthesis, what happens if the rotor fails to spin?}
\end{enumerate}
\noindent To answer these questions, we need to extract the process structure and reason over causal and temporal relations between process events. 
Question answering systems that rely on bag-of-words representations will fail to correctly answer such questions.

\FigStar{figures/process3}{0.8}{process}{An annotation of the ATP synthesis process}

Process extraction is related to two recent lines of work in Information Extraction -- event extraction and timeline construction.
Traditional tasks in event extraction focuses on identifying a small closed set of events in a single sentence. 
For example, the BioNLP 2009 and 2011 shared tasks \cite{kim09,kim11} consider nine events types that are relevant for proteins.
Process extraction, on the other hand, are centered around discovering \emph{relations} between events that span \emph{multiple} sentences. The set of possible event types in process extraction is also much larger.

Timeline construction involves identifying temporal relations between events \cite{Chambers08,Yoshikawa09,Denis11,Do12,Mcclosky12}, and is thus related to process extraction as both focus on event-event relations that span multiple sentences. However, fully capturing process structure requires handling a rich set of relations such as \textsc{Causes} and \textsc{SuperEvent} (see Section~\ref{sec:process}), which are often not addressed in timeline construction. Moreover, events in processes exhibit much higher degrees of dependencies, for example, in general all events in a process are related to one another. This property does not hold in temporal ordering. 

In this paper, we formally define the task of process extraction and present automatic extraction methods. 
Our approach works over multiple sentences and extracts a rich set of event-event relations, where the set of possible event types is open ended. 
Furthermore, we characterize a set of global properties in process structure that can be utilized during process extraction. 
For example, most processes exhibit a ``chain-like" structure corresponding to process progression over time, and all process events are connected to one another, as previously noted. 
We show that by incorporating global properties into our model and performing joint inference over the extracted relations, we can significantly improve process quality.  
Our empirical experiments are performed over a novel data set of 150 process descriptions from the textbook ``Biology" \cite{CampbellReece} that were annotated by trained biologists. Our method does not utilize any domain-specific knowledge and can be easily adopted for domains other than Biology.

The three main contributions of this paper are:
\begin{enumerate}[itemsep=0pt,topsep=0pt] 
\item We define process extraction and characterize the structural properties of processes.
\item We show that modeling global structural properties significantly improves process extraction accuracy.
\item  We publicly release a novel data set of 150 fully annotated biological process descriptions.
\end{enumerate}
