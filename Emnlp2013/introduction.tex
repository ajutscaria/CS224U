\section{Introduction}

\textbf{Motivation:} Being able to reason over processes is crucial for language understanding applications. Consider for example the paragraph in Figure~\ref{fig:process}, which describes the process of ATP synthesis. A human reading this paragraph will be able to answer complex questions that require understanding of the process structure such as

\begin{enumerate}
\item \emph{How do H+ ions contribute to the production of ATP?}
\item \emph{What causes the rotor to spin?}
\item \emph{In ATP synthesis, what happens if the rotor fails to spin?}
\end{enumerate}

\noindent All these questions require understanding and reasoning over processes, and thus systems that have only bag-of-words representations will fail.In this paper we suggest a method for extracting a process structure that will facilitate answering of complex questions.

\FigStar{figures/process}{0.8}{process}{An annotation of the ATP synthesis process}

\textbf{Relation to previous work:} 
Extracting processes is related to two lines of works in Information Extraction - event extraction and timeline construction. Recent work in event event extraction \cite{riedel11fast,Mcclosky11} is based on BioNLP challenges and focuses on extraction of a closed set of events such as \emph{regulation} and \emph{phosphorilation} from a single sentence and their relations to proteins. However, a process is typically described over multiple sentences and involves a large number of possible events. Work on timeline construction \cite{Do12,Mcclosky12} requires partially ordering a set of events that is described in a sequence of sentence. However, fully capturing process structure requires a rich set of relations (\emph{cause}, \emph{super}) that is missing from this line of work.

\textbf{Emphasizing this work:} In this paper, we find the structure of a biological process by extracting the relations between the process events. Properties of our task (a) spans multiple sentences (b) open-ended set of events (c) rich set of relations comparing to timeline construction (d) the nature of the text - it is a textbook rather than abstracts. (e) We do not use domain-specific knowledge. Some sentence that says that by doing this we will be able to answer the complex question from the beginning - linking to language understanding.

\textbf{Technical contribution} Processes have a global structure and we want to take advantage of that when extracting the relations. For example, all events a process description are connected to one another and there are various constraints such as if two event mentions refer to the same event then they must be related in a similar way to a third event. Similar to  many recent works in NLP \cite{} we model global constraints using ILP, however since many of the constraints can be violated we use soft constraints. We show that by encoding global constraints we can substantially improve performance.

\textbf{Contributions} Three main contributions
\begin{itemize}
\item We define the task of process extraction - what is a process - we also identify properties of the process structure
\item We propose a method for process extraction that uses global constraints and show that it improves performance
\item We release a set of 150 biological processes, annotated by biologists.
\end{itemize}

\textbf{structure} Background, Definition and properties of processes, Local classifier (old features and new features), Global model, Experiments and maybe analysis