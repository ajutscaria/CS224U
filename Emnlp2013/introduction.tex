\section{Introduction}

Processes describe complicated phenomena that involve a series of events and multiple participants. Automatically extracting the structure of processes is necessary for text understanding applications that require reasoning over process events. Consider, for example, the paragraph in Figure~\ref{fig:process}, which describes the biological process of ATP synthesis. A human reading this paragraph can create a mental model that allows her to answer questions such as:

\begin{enumerate}[itemsep=0pt,topsep=0pt] 
\item \small \emph{How do H+ ions contribute to the production of ATP?}
\item \small\emph{What causes the rotor to spin?}
\item \small \emph{In ATP synthesis, what happens if the rotor fails to spin?}
\end{enumerate}

\noindent All these questions depend on extraction of the process structure and reasoning over the causal and temporal relations between the process events. Question answering systems that rely on bag-of-words representations will fail to correctly answer such questions.

\FigStar{figures/process}{0.8}{process}{An annotation of the ATP synthesis process}

Process extraction is related to two recent lines of works in Information Extraction -- event extraction and timeline construction. The BioNLP 2009 and 2011 shared tasks \cite{kim09,kim11} led to increasing interest in biomedical event extraction \cite{Poon10,Miwa10,riedel11fast,Mcclosky11,Bjorne11}, where given a single sentence annotated with protein mentions, events  are identified and relations between events and proteins are extracted. In this shared task participants were asked to consider nine event types that are relevant for proteins (such as \emph{Phosphorilation} and \emph{Transcription}). Processes, on the other hand,  are centered around discovering relations between several event mentions. Thus, process descriptions usually span multiple sentences, and must handle both an open-ended set of event types as well as a rich set of event-event relations.

Timeline construction involves identifying temporal relations between a collection of events \cite{Chambers08,Yoshikawa09,Denis11,Do12,Mcclosky12}, and is thus related to process extraction as both focus on event-event relations that span multiple sentences. However, fully capturing process structure requires handling a rich set of relations such as \textsc{Causes} and \textsc{SuperEvent} (see Section~\ref{sec:process}), which are often not addressed in timeline construction. Moreover, processes exhibit particular properties that do not hold generally in temporal ordering. For example, in processes all events are somehow related to one another, a property that can be exploited for improving extraction.

In this paper, we present the task of process extraction and describe methods for extracting relations between process events. Our method works over multiple sentences and extracts a rich set of event-event relations, where the set of possible event types is open ended. Process structure is characterized by global properties that can be utilized during process extraction. For example, most processes exhibit a ``chain-like" structure corresponding to process progression over time, and all process events are connected to one another, as previously noted. We will show that by incorporating global properties into our model and performing joint inference over the extracted relations we can significantly improve process quality.  Our empirical experiments are performed over a novel data set of 150 process descriptions from the textbook ``Biology" \cite{CampbellReece} that were annotated by trained biologists. We note however that our method does not utilize any domain-specific knowledge and thus can be easily applied to domains other than Biology.

To conclude, this paper presents the following three contributions:
\begin{enumerate}[itemsep=0pt,topsep=0pt] 
\item We define the task of process extraction and characterize the structural properties of processes.
\item We show that by modeling structural properties we can significantly improve the quality of extracted processes comparing to several baselines.
\item  We publicly release a novel data set of 150 fully annotated biological process descriptions.
\end{enumerate}
