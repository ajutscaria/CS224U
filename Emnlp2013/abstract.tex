\begin{abstract}
%On a daily basis, we encounter process descriptions detailing step-by-step events that results in an outcome we care about. Likewise, biological processes are complex phenomena involving a series of events that are related to one another through multiple dependencies. Computers that can understand and reason over text describing biological processes have the potential to dramatically improve the performance of semantic applications such as question answering (QA) - specifically answering \emph{"How?"} and \emph{"Why?"} questions. In this paper, we present the task of \emph{process extraction}, in which a set of temporal, causal and co-reference relations between events within a process are automatically extracted from text. We first design a classifier with novel features that extracts relations between every pair of events. We then extend the model to encode global properties of events constituting a process, for example, by ensuring that the graph of events (with relations as edges) in the process is \emph{connected}. Our method performs joint inference over the set of all possible relations and enforces global constraints that characterize structural properties including connectivity and contradiction detection. On a novel biology dataset, containing 148 descriptions of biological processes (released with this paper) we show significant improvement in predicting event relations in comparison to strong baselines that disregard process structure.

Biological processes are complex phenomena involving a series of events that are related to one another through multiple dependencies. Systems that can understand and reason over text describing biological processes could dramatically improve the performance of semantic applications such as question answering (QA) - specifically \emph{``How?"} and \emph{``Why?"} questions. In this paper, we present the task of \emph{process extraction}, in which events within a process and the relations between the events are automatically extracted from text. We represent processes by graphs whose edges describe a large set of temporal, causal and co-reference event-event relations, and characterize the structural properties of these graphs (e.g., the graphs are \emph{connected}). Then, we present a method for extracting relations between the events, which exploits these structural properties by performing joint inference over the set of extracted relations. On a novel dataset containing 148 descriptions of biological processes (released with this paper), we show significant improvement comparing to baselines that disregard process structure.
\end{abstract}
