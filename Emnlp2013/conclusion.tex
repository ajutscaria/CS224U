\section{Conclusion}

Developing systems that understand process descriptions is an important step towards applications that require deeper reasoning, such as building biological process models from text, intelligent tutoring systems, and non-factoid QA systems. In this paper we have presented the task of process extraction, and developed methods for extracting relations between process events. Processes contain events that are  tightly coupled through strong dependencies. We have shown that  exploiting these structural dependencies and performing joint inference over all event mentions can significantly improve accuracy ove several baselines. We have also released a new data set containing 148 fully annotated descriptions of biological processes. Though the models we built were trained on biological processes, they do not encode domain specific information, and hence should be extensible to other domains.

In this paper we assumed that event triggers are given as input. In future work we want to perform trigger identification jointly with extraction of event-event relations. Since data annotation is expensive, an important direction is to reduce the annotation burden by using data from similar domains or unannotated corpora. Lastly, we would like to integrate our method into QA systems and use the extracted process structure to answer non-factoid questions. 
