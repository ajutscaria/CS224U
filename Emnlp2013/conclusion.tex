\section{Conclusion}

Developing systems that understand process descriptions is an important step towards building applications that require deeper reasoning, such as biological process models from text, intelligent tutoring systems, and non-factoid QA systems. In this paper we have presented the task of process extraction, and developed methods for extracting relations between process events. Processes contain events that are  tightly coupled through strong dependencies. We have shown that  exploiting these structural dependencies and performing joint inference over all event mentions can significantly improve accuracy over several baselines. We have also released a new dataset containing 148 fully annotated descriptions of biological processes. Though the models we built were trained on biological processes, they do not encode domain specific information, and hence should be extensible to other domains.

In this paper we assumed that event triggers are given as input. In future work, we want to perform trigger identification jointly with extraction of event-event relations. As explained in Section~\ref{sec:experiment}, the performance of our system is confined by the performance of the local classifier, which is trained on relatively small amounts of data. Since data annotation is expensive, it is important to improve the local classifier without increasing the annotation burden. For example, one can use unsupervised methods that learn narrative chains \cite{Chambers11} to provide some prior on the typical order of events. Alternatively, we can search on the web for redundant descriptions of the same process and use this redundancy to improve classification. Last, we would like to integrate our method into QA systems and allow non-factoid questions that require deeper reasoning to be answered by matching the questions against the learned process structures.