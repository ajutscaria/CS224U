\subsection{Semantic Role Labeling}
For argument identification, we had used a binary classifier based on MaxEnt model that predicts a probabilty value it to be an argument. For semantic role labeling, we extended this model to a multi-class classifier to generate probability values for a parse-tree node to belong to the different semantic roles. We also modified the dynamic program for non-overlapping constraint used in argument prediction to a re-ranking model that jointly assigns semantic role to all nodes in a sub-tree. We also combine the probability values generated by the $P_A$ model while predicting the semantic roles for parse tree nodes. The values that we use correspond to probabilities generated in the last stage of argument prediction in the iterative optimization algorithm. Let $L$ be the semantic role labels that have been assigned to the entities, including $NONE$ if it is not an entity.

$P_{SRL}(L| e, x) = P_A(A|e, x) P_L(L|e,x) $

We use a bottom up re-ranking approach by keeping the top-k joint assignment of semantic roles to all nodes in a sub-tree. This algorithm is similar to the dynamic programming for non-overlapping constraints. At the pre-terminal nodes, we keep just the semantic roles of the word subsumsed by the node in descending order of probability. At nodes above the pre-terminal nodes, there are two scenarios:

\begin{enumerate}
\item The node is an argument to the trigger: In this case, the node has a non-NONE semantic role and none of the children nodes can be entities and hence all children would have semantic role $NONE$. For each non-NONE semantic role of the node, we compute the probability value of the joint labeling of the sub-tree.
\item The node is not an argument to the trigger: In this case, the node has a semantic role of $NONE$. Since at each of the child nodes we have top-k possible assignements, we take all combinations of assignments of children nodes and compute the probability for each of these possible joint labelings.
\end{enumerate}

The next step is to re-rank all the possible joint assignments possible at the node and then retain only the top-k at the node. This algorithm proceeds until we reach the root and the joint labeling that has the highest probability will be the semantic roles predicted by the model.
