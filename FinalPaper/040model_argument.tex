\subsection{Argument identification for triggers}
An entity is represented as a node in the parse tree spanning over the full text of the entity along the leaves of the tree. The fact that there is more than one event in almost all sentences makes our task of event-entity association harder. This is because, instead of just predicting a node in the parse tree as an entity, we have to predict if a node is associated with a specific trigger from task 1. To overcome this, we assign a probability for each node in the parse tree to be an entity associated with each event. If $N$ is the set of $n$ nodes $\{t_1, t_2, ... t_n\}$ in the parse tree, given an event trigger $e$, our goal is to predict the best labeling, $A$ of parse tree nodes to $ARG$ or $NONE$ for the event trigger $e$.

We implemented a MaxEnt based model using more features between the event triggers and the candidate nodes to classify the node as either an $ARG$ or $NONE$ if not. This is also a local model, hence we try to maximize

$P(A | e, x) = \prod_{t_{i}\in N} P(t_{i} \in ARG | e, x) $

The features we use for finding event arguments are combinations of features of both event node and the candidate entity node. This includes POS tag of candidate + POS tag of event trigger, headword of candidate + POS tag of event trigger, path from the candidate to the event trigger and an indicator feature denoting whether the headword of the candidate is a child of the trigger in the dependency tree. The model assigns a probability value for each node in the parse tree to be an argument to a specific event trigger.

{\bf Dynamic Program for non-overlapping constraint.} Since we predict a node in the parse tree as an argument to an event trigger or not, there are instances when predicted entities overlap. For instance, a sub-tree of a tree node already marked as an entity may also be tagged as an entity. To avoid this, we devised a bottom-up dynamic program that tags a node as entity or not, looking at the probability of the node and its immediate children being entities. This is motivated by the work of Toutanova et al.~\shortcite{toutanova}. The dynamic program works from the pre-terminal nodes of the tree and find best assignments for each tree using the already computed assignments for its children. This ensures that a sub-tree that is a part of $A$ in itself doesn't have smaller subtrees that belongs to $A$. The most likely assignment for a tree $t$  to $A$ or $NONE$ is the one that corresponds to the maximum of:

\begin{enumerate} 
\item The sum of the log-probabilities of the most likely assignments of the children sub-trees $t_1, t_2,.. t_k$, plus the log-probability for assigning the node $t$ to $NONE$.
\item The sum of the log-probabilities for assigning all of $t_i$'s nodes to $NONE$ plus the log-probability of assigning the node $t$ to $A$.
\end{enumerate}

Another addition we did to the dynamic program was that an entity node cannot subsume a node in the parse tree that is an event trigger.