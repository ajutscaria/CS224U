
\begin{table}
\centering
\begin{tabular}{|l||c|c|c|} \hline
&\textbf{Precision} & \textbf{Recall} & \textbf{F1} \\ \hline
%%%%%%%%%% NB: I changed the meaning of the sign!!!!!
\hline
Baseline& 0.47 & 0.72&0.57\\
MaxEnt-Train& 0.86 & 0.71&  0.77 \\
MaxEnt-CV&0.71&0.67&0.69\\
\hline
\end{tabular}
\caption{Event trigger prediction}
\label{table:eventprediction}
\end{table}

\begin{table}
\centering
\begin{tabular}{|l||c|c|c|} \hline
&\textbf{Precision} & \textbf{Recall} & \textbf{F1} \\ \hline
%%%%%%%%%% NB: I changed the meaning of the sign!!!!!
\hline
Baseline&0.52&0.70&0.59\\
MaxEnt-Train&0.81&0.66&0.72\\
MaxEnt-CV&0.69&0.60&0.64\\
\hline
\end{tabular}
\caption{Entity prediction for event triggers}
\label{table:entityprediction}
\end{table}

The results for event prediction task are presented in Table~\ref{table:eventprediction}. Baseline for event - This model performed quite well giving an F1 score of 0.565, considering that it was a very naive approach. 

The results are presented in Table~\ref{table:entityprediction}. The baseline model intuitively captures the relation between event triggers and its arguments as is evident from the F1 score of 0.593 achieved using a relatively simple approach. Results after using DP - The dynamic programming approach gave us a boost of 0.04 (0.60 to 0.64) in F1 score. 