Most of the previous work addresses a subset of tasks we focus on in this paper or look at very specific types of events or entities to be extracted.  For instance, Toutanova et al.~\shortcite{toutanova} comes up with a joint model for argument extraction and semantic role labeling assuming that the trigger words are provided while McClosky et al.~\shortcite{mcclosky} focuses on event and argument extraction for only binding, regulation and phosphorylation event types. Bjorne et al.~\shortcite{bjorne} solves the problem of event extraction and semantic role labeling assuming the entities are known. Also, their event categories and arguments are closely tied to the BIONLP task and hence, would not generalize to domain-independent text. Chambers and Jurafsky~\shortcite{chju2009} addresses the area of event identification, and also combines argument extraction with semantic role labeling using an unsupervised approach addressing narrative event chains in further detail. 

Different approaches have been adopted for event and entity extraction. While some of the previous work has used unsupervised learning methods~\cite{chju2008},~\cite{chju2009}, there are others who viewed each task as a separate machine learning problem ~\cite{bjorne} or tried to combine the problems using joint models that can overcome the problem of cascading errors and does not assume independence that separate classifiers for event and entity extraction do~\cite{toutanova},~\cite{riedelmc}.

Events and entities have been represented in different ways; the work of Bjorne et al.~\shortcite{bjorne} and Riedel and McCallum~\shortcite{riedelmc} uses graphs to encode event-entity and event-event relationships and looks at the problem of joint event extraction as a structure prediction problem. Toutanova et al.~\shortcite{toutanova} and McClosky et al.~\shortcite{mcclosky}, use tree structure to represent the nodes and use semantic role labeling to solve this problem. Also, some of the work relies on the constituency parse structure of sentences~\cite{toutanova}, while others~\cite{bjorne},~\cite{mcclosky} use the dependency parse structure for finding features. 

There has also been work on temporal ordering of events thus extracted. The temporal relation between events can be considered as a classification problem, the number of classes depending on the number of relations being considered in the task~\shortcite{chju2008},~\shortcite{temporalordering}. McClosky et al.~\shortcite{mcclosky} uses a timeline model in addition to this that allows events to be associated with precise time intervals, which improves human interpretability and also simplifies the global inference formulations required to solve this task.	