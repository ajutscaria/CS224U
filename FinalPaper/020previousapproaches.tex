Most of the prevoius work dealt only with a subset of the three tasks listed as the project goals and dealt with very specific domains.  For instance, Chambers and Jurafsky ~\shortcite{chju2008} talks about event identification and temporal ordering using narrative event chains.Toutanova comes up with a joint model for argument extraction and semantic role labeling assuming that the trigger words are provided. Bjorne solves the problem of event extraction and semantic role labeling assuming the entities are known. Also, their event categories and arguments are closely tied to the BIONLP task and hence, would not generalize to domain-independent text. Chambers 2009 addresses the areas of event identification, and combines argument extraction with semantic role labeling using an unsupervised approach addressing narrative event chains in further detail. 

Different approaches have been tried to come up with a solution to these problems. While some work has used unsupervised learning methods (Chambers and Jurafsky 2008, 2009), there are others who viewed each task as a separate machine learning problem (Bjorne) or tried to combine the problems using joint models that can overcome the problem of cascading errors and does not have to assume independence of events and enitites (Toutanova and , Riedel and McCallum).

Events and entities have been represented in different ways; the work of Bjorne and McClosky and , Riedel and McCallum use graphs to represent relationship between events and entities, while Toutanova and McClosky, uses tree structure with classification of nodes as an entity or not. Also, some of the work rely on the constituency parse structure of sentences (Toutanova), while others ( Bjorne and McClosky) use the dependency parse structure for finding features. 


